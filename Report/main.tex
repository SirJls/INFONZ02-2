%------------------------------------------------------------------------------
%	PACKAGES AND OTHER DOCUMENT CONFIGURATIONS
%------------------------------------------------------------------------------

\documentclass[twoside,twocolumn]{article}

\usepackage{blindtext} % Package to generate dummy text throughout this template

\usepackage[sc]{mathpazo} % Use the Palatino font
\usepackage[T1]{fontenc} % Use 8-bit encoding that has 256 glyphs
\linespread{1.05} % Line spacing - Palatino needs more space between lines
\usepackage{microtype} % Slightly tweak font spacing for aesthetics

\usepackage[dutch]{babel} % Language hyphenation and typographical rules
\usepackage[titletoc,title]{appendix}

\usepackage[hmarginratio=1:1,top=32mm,columnsep=20pt]{geometry} % Document margins
\usepackage[hang, small,labelfont=bf,up,textfont=it,up]{caption} % Custom captions under/above floats in tables or figures
\usepackage{booktabs} % Horizontal rules in tables

\usepackage{lettrine} % The lettrine is the first enlarged letter at the beginning of the text

\usepackage{enumitem} % Customized lists
\setlist[itemize]{noitemsep} % Make itemize lists more compact

\usepackage{abstract} % Allows abstract customization
\renewcommand{\abstractnamefont}{\normalfont\bfseries} % Set the "Abstract" text to bold
\renewcommand{\abstracttextfont}{\normalfont\small\itshape} % Set the abstract itself to small italic text

\usepackage{titlesec} % Allows customization of titles
\renewcommand\thesection{\Roman{section}} % Roman numerals for the sections
\renewcommand\thesubsection{\roman{subsection}} % roman numerals for subsections
\titleformat{\section}[block]{\large\scshape\centering}{\thesection.}{1em}{} % Change the look of the section titles
\titleformat{\subsection}[block]{\large}{\thesubsection.}{1em}{} % Change the look of the section titles

\usepackage{fancyhdr} % Headers and footers
\pagestyle{fancy} % All pages have headers and footers
\fancyhead{} % Blank out the default header
\fancyfoot{} % Blank out the default footer
\fancyhead[C]{Running title $\bullet$ May 2016 $\bullet$ Vol. XXI, No. 1} % Custom header text
\fancyfoot[RO,LE]{\thepage} % Custom footer text

\usepackage{titling} % Customizing the title section

\addto\captionsdutch{\renewcommand{\appendixname}{Bijlage}} % changes 'Bÿlage' to 'Bijlage' for the bookmarks

\usepackage[
     pdfusetitle
    ,colorlinks=true
    ,citecolor=link
    ,linkcolor=link
    ,pdfborder={0 0 0}
    ,unicode=true
]{hyperref}

%------------------------------------------------------------------------------
%	TITLE SECTION
%------------------------------------------------------------------------------

\setlength{\droptitle}{-4\baselineskip} % Move the title up

\pretitle{\begin{center}\Huge\bfseries} % Article title formatting
\posttitle{\end{center}} % Article title closing formatting
\title{Accuraatheid van het aanleren van functioneel programmeren aan studenten
door het gebruikt van Parser Combinators. } % Article title
\author{%
\textsc{Sjors Sparreboom} \\[1ex] % Your name
\normalsize Hogeschool Rotterdam \\ % Your institution
\normalsize \href{mailto:0890040@hr.nl}{0890040@hr.nl} \\[2ex] % Your email address
%\and % Uncomment if 2 authors are required, duplicate these 4 lines if more
\textsc{Johan Bos} \\[1ex] % Second author's name
\normalsize Hogeschool Rotterdam \\ % Second author's institution
\normalsize \href{mailto:0878090@hr.nl}{0878090@hr.nl} % Second author's email address
}
\date{\today} % Leave empty to omit a date
\renewcommand{\maketitlehookd}{%
% TODO: Improve abstract <20-11-17, SirJls> %
\begin{abstract}
\noindent
   Rapportage over het onderzoeken, hoe parser combinators bij kunnen dragen
   aan het aanleren van functioneel programmeren aan studenten. Ook wordt er
   gekeken hoe of parser combinators een te overwegen manier zijn om moeilijke
  concepten zoals "moands" en "applicatives" aan te leren.
\end{abstract}
}

%------------------------------------------------------------------------------

\begin{document}
\nocite{*}

% Print the title
\maketitle

%------------------------------------------------------------------------------
%	ARTICLE CONTENTS
%------------------------------------------------------------------------------

\section{Inleiding}
\lettrine[nindent=0em,lines=3]{I} n de wereld van functioneel programmeren,
zijn parser combinators, al enige tijd bekend als een uitstekende manier voor
de ontwikkeling van parsers, compilers en domein specifieke talen. Zo zijn
parser combinators opgebouwd als kleine functies die vervolgens door
compositie een geheel nieuwe combinator of parser geven. Dit is erg krachtig
aangezien grote veranderingen in de parsers makkelijk aangebracht kunnen
worden, wat ze zeer geschikt maar voor prototypen van parsers en compilers.

Door middel van dit onderzoek wordt er gekeken of de toepassing van parser
combinators breder zijn dan alleen het prototypen van parsers en compilers.
Aantonend dat de theorie van functioneel programmeren, duidelijk aangeleerd kan
worden aan studenten door ze bekend te maken met de theorie achter parser
combinators. Vervolgens zal er een klein begin gemaakt worden met een parser
combinator bibliotheek die gemakkelijk uitbreidbaar is voor meer complexere
grammatica. Deze bibliotheek kan ook gebruikt worden als referentie materiaal
voor docenten om studenten hun eigen parser combinator bibliotheek te laten
implementeren.

%------------------------------------------------------------------------------

\section{Generieke Termen}
parser combinators, deterministic, parser generators, lazy evaluation.

%------------------------------------------------------------------------------

\section{Methode}

\subsection{Informatie Winning Proces}
Om relevante studie materiaal te identificeren is er voor een multimodale
zoek strategie gekozen, waarbij zowel elektronische als manuele zoekopdrachten
werden verricht. Het literatuurzoekprocess bestaat uit twee fases. In de eerste
fase gaat het over verwezenlijken van voorafgaande zoekresultaten, om de
onderzoeksvraag the kunnen bijstellen en de sleutel concepten te kunnen
bepalen. In deze fase hebben wij onze onderzoeksvraag bijgesteld om ervoor te
zorgen dat functioneel programmeren werd inbegrepen. De hoofd zoekresultaten
werden verkregen in de tweede fase (November, 2017).

Elektronische zoekopdrachten zijn uitgevoerd, gebruikmakend van de volgende
zoekmachines: Google Scholar, Research Gate, Academic Microsoft, CiteSeerX.
Onder de zoektermen die werden gebruikt om deze bibliografische databanken te
doorzoeken waren  "functional programming", "parser combinators",
"teaching functional programming" en "parsing". Voor een volledige lijst van
zoektermen zie \cref{appendix:keywords}.

% TODO: Explain the procedure about how we got the knowledge <20-11-17, SirJls> %
\subsection{Procedure}
\blindtext

\subsection{Inclusie Criteria \& Exclusie Criteria}
\textbf{Algemene criteria} \quad Voor het identificeren van onderzoeken die
de 'accuraatheid van het aanleren van functionele programmeertalen door parser
combinators', rapporteren, werd er gezocht naar 'parser combinators' en
'studenten'. Onderzoeken die de accuraatheid van het aanleren van
functioneel programmeren meten, anders dan door gebruik van parser combinators
zijn uitgesloten.

% TODO: Proceed with the inclusion material

%------------------------------------------------------------------------------

\section{Resultaten}

\blindtext % Dummy text

\blindtext % Dummy text

%------------------------------------------------------------------------------

% TODO: Invoke a discussion and associate conclusion inside disussion <20-11-17, SirJls> %

\section{Discussion}

\blindtext % Dummy text

%------------------------------------------------------------------------------
%	REFERENCE LIST
%------------------------------------------------------------------------------

\bibliographystyle{IEEEtran}
\bibliography{main}

\clearpage

%------------------------------------------------------------------------------
%	APPENDIX
%------------------------------------------------------------------------------
\appendix

% TODO: Fill in keywords inside appendix <20-11-17, SirJls> %
\begin{appendices}
  \section{Zoektermen}
  \label{appendix:keywords}
  De volgende zoektermen waren ingevoerd in de elektronische databanken:

  \blindtext % Dummy text
\end{appendices}

%------------------------------------------------------------------------------

\end{document}
