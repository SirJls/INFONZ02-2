%------------------------------------------------------------------------------
%	PACKAGES AND OTHER DOCUMENT CONFIGURATIONS
%------------------------------------------------------------------------------

\documentclass[twoside,twocolumn]{article}

\usepackage{blindtext} % Package to generate dummy text throughout this template

\usepackage[sc]{mathpazo} % Use the Palatino font
\usepackage[T1]{fontenc} % Use 8-bit encoding that has 256 glyphs
\linespread{1.05} % Line spacing - Palatino needs more space between lines
\usepackage{microtype} % Slightly tweak font spacing for aesthetics

\usepackage[dutch]{babel} % Language hyphenation and typographical rules
\usepackage[titletoc,title]{appendix}

\usepackage[hmarginratio=1:1,top=32mm,columnsep=20pt]{geometry} % Document margins
\usepackage[hang, small,labelfont=bf,up,textfont=it,up]{caption} % Custom captions under/above floats in tables or figures
\usepackage{booktabs} % Horizontal rules in tables

\usepackage{lettrine} % The lettrine is the first enlarged letter at the beginning of the text

\usepackage{enumitem} % Customized lists
\setlist[itemize]{noitemsep} % Make itemize lists more compact

\usepackage{abstract} % Allows abstract customization
\renewcommand{\abstractnamefont}{\normalfont\bfseries} % Set the "Abstract" text to bold
\renewcommand{\abstracttextfont}{\normalfont\small\itshape} % Set the abstract itself to small italic text

\usepackage{titlesec} % Allows customization of titles
\renewcommand\thesection{\Roman{section}} % Roman numerals for the sections
\renewcommand\thesubsection{\roman{subsection}} % roman numerals for subsections
\titleformat{\section}[block]{\large\scshape\centering}{\thesection.}{1em}{} % Change the look of the section titles
\titleformat{\subsection}[block]{\large}{\thesubsection.}{1em}{} % Change the look of the section titles

\usepackage{fancyhdr} % Headers and footers
\pagestyle{fancy} % All pages have headers and footers
\fancyhead{} % Blank out the default header
\fancyfoot{} % Blank out the default footer
\fancyhead[C]{Het functionele paradigma is de toekomst} % Custom header text
\fancyfoot[RO,LE]{\thepage} % Custom footer text

\usepackage{titling} % Customizing the title section

\addto\captionsdutch{\renewcommand{\appendixname}{Bijlage}} % changes 'Bÿlage' to 'Bijlage' for the bookmarks

\usepackage[
     pdfusetitle
    ,colorlinks=true
    ,citecolor=link
    ,linkcolor=link
    ,pdfborder={0 0 0}
    ,unicode=true
]{hyperref}

%------------------------------------------------------------------------------
%	TITLE SECTION
%------------------------------------------------------------------------------

\setlength{\droptitle}{-4\baselineskip} % Move the title up

\pretitle{\begin{center}\Huge\bfseries} % Article title formatting
\posttitle{\end{center}} % Article title closing formatting
\title{Het functionele paradigma is de toekomst.} % Article title
\author{%
\textsc{Sjors Sparreboom} \\[1ex] % Your name
\normalsize Hogeschool Rotterdam \\ % Your institution
\normalsize \href{mailto:0890040@hr.nl}{0890040@hr.nl} \\[2ex] % Your email address
%\and % Uncomment if 2 authors are required, duplicate these 4 lines if more
\textsc{Johan Bos} \\[1ex] % Second author's name
\normalsize Hogeschool Rotterdam \\ % Second author's institution
\normalsize \href{mailto:0878090@hr.nl}{0878090@hr.nl} % Second author's email address
}
\date{\today} % Leave empty to omit a date
\renewcommand{\maketitlehookd}{%
\begin{abstract}
\noindent
  Met de komst van het internet, big data, cloud computing en de
  toenemende capaciteit van computers is er vernieuwde interesse in de
  functionele programmeertalen. Dit rapport bekijkt samen met de lezer de voor
  en nadelen bij de stelling
  \textit{``Het functionele paradigma is de toekomst.''} Dit gebeurt op basis
  van een literatuuronderzoek naar de relaties en verschillen van de heersende
  programmeerparadigma's in de industrie. Het onderzoek beperkt zich tot het
  imperatieve en functionele paradigma. Om het functionele paradigma en zijn
  plaats binnen de informatica beter te kunnen begrijpen is er binnen dit
  onderzoek van een minimale kennisgeving van de heersende
  programmeerparadigma's in de industrie uitgegaan. Ook is kennis van
  programmeren wenselijk, maar kan het onderzoek ook prima zonder deze kennis
  worden gevolgd.
\end{abstract}
}

%------------------------------------------------------------------------------

\begin{document}
\nocite{*}

% Print the title
\maketitle

%------------------------------------------------------------------------------
%	ARTICLE CONTENTS
%------------------------------------------------------------------------------

\section{Inleiding}
\lettrine[nindent=0em,lines=3]{V}oor het maken van programma's wordt er gebruik
gemaakt van programmeertalen.  Door het gebruik van een programmeertaal is het
mogelijk voor een programmeur om een computer te voorzien van instructies en zo
de gigantische rekenkracht van het apparaat te benutten. Er zijn door de jaren
heen veel programmeertalen ontwikkelt. Sommige revolutionair en vernieuwend,
andere een uitbreiding op reeds bestaande talen. Veel van deze programmeertalen
delen dezelfde kenmerken en kunnen worden geclassificeerd onder een
paradigma.

Het is daarom van essentieel belang dat een keuze voor een paradigma
afhankelijk is van een correcte probleem- of doelstelling. Zo wordt er vanuit
de academische en wetenschappelijke hoek al jaren aangezet tot een transitie
naar het functionele paradigma, maar worden functionele programmeertalen, zoals
Haskell en Lisp vaak als een onconventionele, onwerkbare vorm van software
ontwikkeling gezien. Echter wordt er momenteel met de komst van big data en
cloud computing een nieuwe effici\"ente manier van dataverwerking toegepast die
veelal gebruik maakt van parallel, gelijktijdig en gedistribueerd programmeren.

Deze oplossingen schikken zich vaak voor een functionele aanpak van het
probleem, maar de achterliggende infrastructuur is vaak imperatief en
objectgeori\"enteerd waardoor een combinatie veelal uitgesloten is, maar wel
wenselijk.

%------------------------------------------------------------------------------

\section{Middenstuk}

In dit middenstuk zal er een korte uitleg worden gegeven wat de imperatieve en
functionele programmeerparadigma's inhouden. Opvolgend zal er worden gekeken
naar twee kanten van deze stelling, namelijk de voor en tegenargumentatie.

Om de plaats van het functionele programmeerparadigma beter te kunnen
begrijpen, zal er eerst gekeken moeten worden naar de andere dominante
paradigma's in de industrie. De paradigma's die kort besproken zullen worden
zijn: het imperatieve en functionele programmeerparadigma.

%------------------------------------------------------------------------------

\subsection{Imperatief programmeren}
Imperatief programmeren houdt in dat een programma een structuur beschrijft
volgens het Turingmodel. Dit programma wordt door de computer ingelezen in het
geheugen om het in kleine delen sequentieel uit te voeren. De tussentijdse
berekeningen worden in een imperatieve taal als data opgeslagen in het
geheugen. Deze data kan vervolgens worden opgehaald en gemuteerd door elk
volgende instructie die toegang heeft tot deze data, totdat er een antwoord
wordt gepresenteerd. Imperatieve programmeertalen zijn door deze eigenschap om
de huidige staat van het programma aan te passen zeer geschikt voor het
programmeren van computer hardware. Zo zijn CPU's imperatieve executie machines
en compilers vertalen direct naar deze taal. Dit is de reden dat imperatief
programmeren hedendaags de gangbare manier van programmeren is.

\subsection{Functioneel programmeren }
Functioneel programmeren is een paradigma dat niet nieuw is, maar al sinds de
jaren 40 bestaat. Het is een paradigma wat is ontstaan uit het
lambda calculus, wat Turingvolledig is. Kenmerkend voor functioneel
programmeren is het programmeren met functies, het vermijden van globale staat,
de niet muteerbare data structuren en een executie model waarin de evaluatie
volgorde niet uitmaakt.

%------------------------------------------------------------------------------

\section{Algemene voordelen functioneel programmeren}

Nu er kort uitleg is gegeven over imperatieve en functionele talen kunnen we
kijken naar de verschillen, in het bijzonder de voor en nadelen. Om de
verschillen echter goed te begrijpen is het handig om te weten dat op het
moment van schrijven van dit rapport het dominante paradigma in de industrie de
imperatieve manier van programmeren is. Het is echter pas sinds kort dat er
hernieuwde interesse is ontstaan in de functionele talen vanwege de
expressiviteit en het vermogen van deze talen om foutloos gelijktijdig
programma's uit te voeren. Deze eigenschappen sluiten heel goed aan bij de
belangen van veel bedrijven die met de komst van het internet,
cloud computing en big data over data beschikken die zo omvangrijk is dat
de traditionele data verwerkingssoftware niet toereikend genoeg is om hiermee
om te gaan. De oplossing: het verdelen van de data over verschillende
multi-core processoren, zodat er betere prestaties kunnen worden behaald en de
verwerking sneller verloopt. Dit heet in de informatica gedistribueerd
programmeren.

Gedistribueerd programmeren is een complex onderwerp in de informatica en niet
zonder enige risico's. Doordat er meerdere multi-core processoren aangesproken
worden, wat tevens gelijktijdig kan gebeuren, moet er goed nagedacht worden
over de evaluatie volgorde. Doordat de aangesproken processoren niet dezelfde
throughput hebben zijn uitspraken over ruimte en tijd complexiteit moeilijk
voor een programmeur. Zeker als we er vanuit kunnen gaan dat het meest
gebruikte paradigma in de industrie de imperatieve stijl van programmeren is.
Het imperatief programmeren is inherent sequentieel en data is muteerbaar wat
het correct uitvoeren van een programma nog meer bemoeilijkt door eventuele
data races. Om data races te voorkomen zijn er in vele imperatieve
programmeertalen locks of mutexes ge\"introduceerd met als doel het correct
afhandelen van de executievolgorde van een imperatief programma. Echter zijn de
ge\"introduceerde oplossingen niet zonder bijwerkingen, denk aan de deadlock.

Functionele programmeertalen hebben echter veel van deze problemen niet. Zo
zijn functionele talen niet afhankelijk van de evaluatie volgorde en zijn er
geen locks of mutexes, wat ze uitsluit voor deadlocks. Dit maakt het
mogelijk om verschillende stukken van een expressie parallel te evalueren,
wat een pre is voor gedistribueerd programmeren. Tevens is de data in de
functionele programmeertalen niet muteerbaar, wat inhoudt dat er geen
bijwerkingen kunnen ontstaan, waardoor data races niet tot nauwelijks voor
kunnen komen in deze talen.

De functionele programmeertalen zijn ook erg schaalbaar, zo zijn bepaalde
algoritmes in functionele programmeertaal vaak minder regels code dan in een
imperatieve taal, doordat functionele talen vaak een hogere abstractie
hanteren, denk hierbij aan het gebruik van hogere graads functies. Dit zorgt
voor een duidelijker en beter herbruikbaar systeem, waar elk stuk code een
eigen, gescheiden verantwoordelijkheid heeft, dit heet modulariteit.

%------------------------------------------------------------------------------

\section{Algemene nadelen functioneel programmeren}
Dat functioneel programmeren tegenwoordig voor steeds meer situaties en
problemen een gangbare oplossing is, hebben we besproken. Echter zijn er ook
situaties en problemen die zich niet schikken voor een functionele oplossing.
E\a'en van deze situaties is als er programma's (ook wel firmware genoemd)
moeten worden geschreven voor embedded systemen. Embedded systemen zijn vaak
gemaakt voor het uitvoeren van specifieke taken, dit in tegenstelling tot
een computer voor algemeen gebruik. Doordat embedded systemen veelal gemaakt
zijn voor het uitvoeren van specifieke taken zijn er veel meer
processorarchitecturen. Deze grote verschillen in processorarchitectuur en de
beperkte interne geheugen die deze chips vaak hebben, maakt ze niet erg
geschikt voor het schrijven van firmware in een functionele programmeertaal,
aangezien functionele talen meer geheugen gebruiken dan andere paradigma's.

Ook de garbage collector van de functionele talen kan worden gezien als een
enorme beperking aangezien de programmeur het onderliggende geheugen niet kan
controleren. Zo zijn de meeste functionele programmeertalen aanzienlijk
langzamer zonder een goed geoptimaliseerde compiler ten opzichte van
imperatieve programmeertalen. Veel van dit verlies in snelheid kan worden
verklaard door de focus op abstractie van de functionele talen. Door dit hogere
niveau van programmeren duurt een translatie naar machine code door de compiler
vaak langer dan de translatie van imperatieve talen.

\section{Conclusie}
\blindtext % Dummy text

%------------------------------------------------------------------------------
%	REFERENCE LIST
%------------------------------------------------------------------------------

\bibliographystyle{IEEEtran}
\bibliography{main}

\clearpage

%------------------------------------------------------------------------------
%	APPENDIX
%------------------------------------------------------------------------------
\appendix

\begin{appendices}
\end{appendices}

%------------------------------------------------------------------------------

\end{document}
